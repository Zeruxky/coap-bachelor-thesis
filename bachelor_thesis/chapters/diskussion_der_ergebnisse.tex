\section{Diskussion der Messergebnisse}
\label{sec:diskussion-der-messergebnisse}

Betrachtet man die dargelegten Messergebnisse in Kapitel \ref{sec:messung}, sieht man, dass bei großen und aufwendigen I/O-Operationen Asynchronität besser abschneidet. Im Gegensatz dazu schlagen sich synchrone Methoden besser, wenn sich Menge der Daten im reservierten Speicher, also kein Nachladen oder Anforderung von weiteren Daten, sofort verarbeiten lässt. Dies lässt sich damit argumentieren, dass, wenn sich alle Daten im Speicher befinden, die synchrone Methode / Programm die Daten ohne weiteren Aufwand verwenden kann. Hingegen bei asynchronen Methoden ist es unvorteilhaft, wenn die Daten sehr klein sind und sich somit der Mehraufwand zum Aufbau der dafür benötigten Zustandsmaschine nicht lohnt.

Dies hat sich in beiden Messszenarien gezeigt, jedoch erst ab einer sehr großen Menge an Daten. Somit profitiert CoAP von Asynchronität, wenn eine große Menge an Daten verarbeitet werden müssen. Ist jedoch die zu verarbeitende Menge an Daten klein, überwiegt der Mehraufwand der asynchronen Zustandsmaschine. Somit wirkt sich die Asynchronität bei kleineren Datenmengen, in diesem Messszenario kleiner als 100000 Bytes, negativ auf die Performanz der CoAP-Bibliothek.

Um diesen negativen Effekt entgegenzuwirken, gehen auch viele Entwickler von Softwareprogrammen, die gewisse asynchrone Methoden anbieten bzw. verwenden, dazu über, anhand von bestimmten Bedingungen oder Kriterien zu ermitteln, ob eine asynchrone oder eine synchrone Variante der zu implementierenden Funktion verwendet werden soll. Damit wird versucht eine asynchrone Methode dem Entwickler zur Verfügung zu stellen, die optimiert auf die jeweiligen Parameter ist und somit in jeglichen Fall die bestmögliche Leistung erbringt.

Jedoch muss der Entwickler abwägen, ob Nachrichten mit solch einer großen Payload innerhalb der Softwareapplikation zur Norm gehören. Anzumerken ist, dass sowohl die synchrone als auch asynchrone Implementierung Raum für Optimierungen offen lässt. Diese sind auch Gegenstand von weiteren Maßnahmen, die man im Rahmen dieses Projekts vornehmen kann. Auf diese werden jedoch im Kapitel \ref{sec:schlussfolgerung} näher eingegangen.