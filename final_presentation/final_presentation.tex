\documentclass[11pt,t,usepdftitle=false,aspectratio=169,usenames,dvipsnames]{beamer}
\usetheme[nototalframenumber,foot,logo,nosectiontitlepage]{uibk}
\headerimage{1}

\usepackage[ngerman]{babel}
\usepackage{lmodern}
\usepackage{xcolor}
\usepackage{minted}
\usepackage{tikz}
\usepackage{hyperref}
\usepackage{csquotes}
\usepackage{pgfplots}
\usepackage{pgfplotstable}

\MakeOuterQuote{"}

\usetikzlibrary{calc}
\usetikzlibrary{positioning}

\title[Finale Präsentation CoAP.NET]{Increasing throughput of server applications by using asynchronous techniques}
\subtitle{A case study on CoAP.NET}
\author{Philip Wille\\Betreuer: Michael Felderer, Andreas Danek}
\date{2022-06-21}

\begin{document}
    \maketitle

    \section{Programmierparadigmen}
    \label{sec:programmierparadigmen}
    \begin{frame}
        \frametitle{Programmierparadigmen}
        \begin{itemize}
            \item<1-> Synchron
            \begin{itemize}
                \item<3-> \textcolor{uibkblue}{\textbf{Stoppt}} den Programmfluss.
                \item<5-> Überprüft periodisch, ob Funktion beendet ist.
                \item<7-> Ist als "Blocked" oder "Waiting" markiert.
            \end{itemize}
            \item<2-> Asynchron
            \begin{itemize}
                \item<4-> Kann im Programmfluss weiter gehen.
                \item<6-> Ein Event markiert die Beendigung der Funktion.
                \item<8-> Ist frei für andere Aufgaben.
            \end{itemize}
        \end{itemize}
    \end{frame}

    \section{Synchroner und asynchroner Server}
    \label{sec:synchroner-und-asynchroner-server}
    \subsection{Synchroner Server}
    \label{subsec:synchroner-server}
    \begin{frame}
        \frametitle{Synchroner Server}
        \begin{figure}[ht]
            \begin{tikzpicture}
                \node at (0,.3) {Database};
                \node at (-3,.3) {Server};
                \node at (-6,.3) {Client 2};
                \node at (-9,.3) {Client 1};
                
                \draw[thick, uibkorange] (-9,0) -- node[left] {prepare}(-9,-0.5);
                \draw[thick, uibkgraym] (-6,0) -- (-6,-0.5);
                \draw[thick, uibkgraym] (-3,0) -- (-3,-0.5);
                \draw[thick, uibkgraym] (-0,0) -- (0,-0.5);
                \onslide<1->
                \draw[->, thick, uibkorange] (-9,-0.5) -- node[midway,above] {Request 1} (-3,-0.5);
                \onslide<2->
                \draw[thick, uibkorange] (-3,-0.5) -- node[left] {prepare SQL}(-3,-1);
                \onslide<3->
                \draw[->, thick, uibkorange] (-3,-1) -- node[midway,above] {SQL Request 1} (-0,-1);
                \draw[thick, uibkgraym] (0, -0.5) -- (0, -1);
                \onslide<4->
                \draw[thick, uibkorange] (-0,-1) -- node [right] {execute} (0,-3);
                \onslide<5->
                \draw[thick, uibkgraym] (-6, -0.5) -- (-6, -1);
                \draw[thick, uibkblue] (-6, -1) -- node[left] {prepare} (-6, -1.5);
                \onslide<6->
                \draw[->, thick, uibkblue] (-6,-1.5) -- node[midway,above] {Request 2} (-3,-1.5);
                \draw[thick, uibkgraym] (-3,-1) -- (-3,-1.5);
                \onslide<7->
                \draw[<-, thick, uibkorange] (-3,-3) -- node [midway, above] {SQL Response 1} (0,-3);
                \draw[thick, uibkgraym] (-3,-1.5) -- (-3,-3);
                \onslide<8->
                \draw[thick, uibkorange] (-3,-3) -- (-3,-3.5);
                \draw[<-, thick, uibkorange] (-9,-3.5) -- node [midway, above] {Response 1} (-3,-3.5);
                \draw[thick, uibkgraym] (-9,-0.5) -- (-9,-3.5);
                \onslide<9->
                \draw[thick, uibkorange] (-9,-3.5) -- node [midway, left] {process} (-9,-5.5);
                \onslide<10->
                \draw[thick, uibkblue] (-3,-3.5) -- node [midway, left] {prepare SQL} (-3,-4);
                \onslide<11->
                \draw[thick, uibkgraym] (-0,-3) -- (-0,-4);
                \draw[->, thick, uibkblue] (-3,-4) -- node [midway, above] {SQL Request 2} (-0,-4);
                \onslide<12->
                \draw[thick, uibkblue] (-0,-4) -- node [midway, right] {execute} (-0,-5);
                \onslide<13->
                \draw[thick, uibkgraym] (-3,-4) -- (-3,-5);
                \draw[<-, thick, uibkblue] (-3,-5) -- node [midway, above] {SQL Response 2} (-0,-5);
                \onslide<14->
                \draw[thick, uibkblue] (-3,-5) -- (-3,-5.5);
                \onslide<15->
                \draw[thick, uibkgraym] (-6,-1.5) -- (-6,-5.5);
                \draw[<-, thick, uibkblue] (-6,-5.5) -- node [midway, above] {Response 2} (-3,-5.5);
                \draw[thick, uibkgraym] (0, -5) -- (0, -5.5);
            \end{tikzpicture}
            \caption{Sequenzdiagramm eines synchronen Servers}
            \label{figure:sequence-diagram-of-synchronous-server}
        \end{figure}
    \end{frame}

    \subsection{Asynchroner Server}
    \label{subsec:asynchroner-server}
    \begin{frame}
        \frametitle{Asynchroner Server}
        \begin{figure}[ht]
            \begin{tikzpicture}
                \node at (0,.3) {Database};
                \node at (-3,.3) {Server};
                \node at (-6,.3) {Client 2};
                \node at (-9,.3) {Client 1};
                
                \draw[thick, uibkorange] (-9,0) -- node[left] {prepare}(-9,-0.5);
                \draw[thick, uibkgraym] (-6,0) -- (-6,-0.5);
                \draw[thick, uibkgraym] (-3,0) -- (-3,-0.5);
                \draw[thick, uibkgraym] (-0,0) -- (0,-0.5);
                \onslide<1->
                \draw[->, thick, uibkorange] (-9,-0.5) -- node[midway,above] {Request 1} (-3,-0.5);
                \onslide<2->
                \draw[thick, uibkorange] (-3,-0.5) -- node[left] {prepare SQL}(-3,-1);
                \onslide<3->
                \draw[thick, uibkgraym] (-6,-0.5) -- (-6,-1.5);
                \draw[->, thick, uibkorange] (-3,-1) -- node[midway,above] {SQL Request 1} (-0,-1);
                \draw[thick, uibkgraym] (0, -0.5) -- (0, -1);
                \onslide<4->
                \draw[thick, uibkorange] (-0,-1) -- node [right] {execute} (0,-3);
                \onslide<5->
                \draw[thick, uibkgraym] (-3,-1) -- (-3,-1.5);
                \draw[->, thick, uibkblue] (-6,-1.5) -- node[midway,above] {Request 2} (-3,-1.5);
                \onslide<6->
                \draw[thick, uibkblue] (-3,-1.5) -- node[midway, left] {prepare SQL} (-3,-2);
                \onslide<7->
                \draw[->, thick, uibkblue] (-3,-2) -- node [midway, above] {SQL Request 2} (-0,-2);
                \onslide<8->
                \draw[thick, uibkgraym] (-3,-2) -- (-3,-3);
                \draw[<-, thick, uibkorange] (-3,-3) -- node [midway, above] {SQL Response 1} (0,-3);
                \onslide<9->
                \draw[thick, uibkorange] (-3,-3) -- (-3,-3.5);
                \draw[<-, thick, uibkorange] (-9,-3.5) -- node [midway, above] {Response 1} (-3,-3.5);
                \draw[thick, uibkgraym] (-9,-0.5) -- (-9,-3.5);
                \onslide<10->
                \draw[thick, uibkblue] (-0,-3) -- node [midway, right] {execute} (-0,-4);
                \onslide<11->
                \draw[thick, uibkorange] (-9,-3.5) -- node [midway, left] {process} (-9,-5.5);
                \onslide<12->
                \draw[thick, uibkgraym] (-3,-3.5) -- (-3,-4);
                \draw[<-, thick, uibkblue] (-3,-4) -- node [midway, above] {SQL Response 2} (-0,-4);
                \onslide<13->
                \draw[thick, uibkblue] (-3,-4) -- (-3,-4.5);
                \draw[thick, uibkgraym] (0,-4) -- (0,-5.5);
                \draw[<-, thick, uibkblue] (-6,-4.5) -- node [midway, above] {Response 2} (-3,-4.5);
                \draw[thick, uibkgraym] (-6,-1.5) -- (-6,-4.5);
                \onslide<14->
                \draw[thick, uibkblue] (-6,-4.5) -- node [midway, left] {process} (-6,-5.5);
                \draw[thick, uibkgraym] (-3,-4.5) -- (-3,-5.5);
            \end{tikzpicture}
            \caption{Sequenzdiagramm eines asynchronen Servers}
            \label{figure:sequence-diagram-of-asynchronous-server}
        \end{figure}
    \end{frame}

    \section{Constrained Application Protocol (CoAP)}
    \begin{frame}
        \frametitle{Constrained Application Protocol (CoAP)}
        \begin{itemize}
            \item<1-> Definiert im \href{https://tools.ietf.org/html/rfc7252}{RFC 7252}.
            \item<2-> Entwickelt für \textcolor{uibkblue}{\textbf{eingeschränkte}} Umgebungen.
            \item<3-> Basiert auf einem \textcolor{uibkblue}{\textbf{Anfragen-Antwort}}-Modell zur Kommunikation.
            \item<4-> Verwendung von \textcolor{uibkblue}{\textbf{U}}ser \textcolor{uibkblue}{\textbf{D}}atagram \textcolor{uibkblue}{\textbf{P}}rotocol (UDP).
            \item<5-> Alternativen: MQTT, CoAP over TCP, CoAP over Websockets.
            \item<6-> Implementierungen für verschiedene Programmiersprachen.
        \end{itemize}
    \end{frame}

    \subsection{CoAP.NET}
    \begin{frame}
        \frametitle{CoAP.NET}
        \begin{itemize}
            \item<1-> Implementierung von CoAP für C\# nach RFC 7252.
            \item<2-> Entwicklung inaktiv.
            \item<3-> Nur teilweise asynchron.
            \item<4-> Bietet eine Implementation für Client und Server.
        \end{itemize}
    \end{frame}

    \section{Bachelorarbeit}
    \begin{frame}
        \frametitle{Bachelorarbeit}

        \begin{itemize}
            \item<1-> \textcolor{uibkblue}{\textit{>>Hat die asynchrone Implementation eines Servers einen Einfluss auf dessen Durchsatzrate?<<}} \onslide<2-> \textbf{- \textcolor{uibkblue}{\textbf{Nein}}}.
            \item<3-> Vollständige Neuimplementierung von CoAP.NET \onslide<4->\textbf{\textcolor{uibkblue}{ausgenommen retransmission und block-wise transfer}}.
            \item<5-> Implementierungen von Tests zur Messung des Durchsatzes.
            \item<6-> Vergleich zwischen synchroner und asynchroner Version.
            \item<7-> Quellcode frei verfügbar auf \href{https://github.com/world-direct/CoAP.NET}{GitHub}.
            \item<8-> In Zusammenarbeit mit World-Direct eBusiness solutions GmbH.
            \item<9-> Fokussierung auf Serialisierung und Deserialisierung als wichtigsten Baustein.
            \item<10-> Implementation von CoAP over TCP zwecks Limitierung von UDP-Packets.
        \end{itemize}
    \end{frame}

    \section{Implementierung}
    \begin{frame}
        \frametitle{Implementierung}

        \begin{figure}[ht]
            \centering
            \begin{tikzpicture}[squarednode/.style={rectangle, draw=uibkblue!60, fill=uibkgraym!5, very thick, minimum size=5mm}]
                \node[squarednode] (udp-transport) {UdpTransport};
                \node[squarednode] (tcp-transport) [below=of udp-transport]{TcpTransport};
                \node[squarednode] (udp-channel) [right=of udp-transport]{UdpChannel};
                \node[squarednode] (tcp-channel) [right=of tcp-transport]{TcpChannel};
        
                \node[squarednode] (serializer) [below right=0.3 of udp-channel] {Serializer};
                \node[squarednode] (request-handler) [right=3 of udp-channel] {RequestHandler};
                \node[squarednode] (response-handler) [right=3 of tcp-channel] {ResponseHandler};
        
                \node[squarednode] (router) [above=of request-handler] {Router};
                \node[squarednode] (resource) [right=of request-handler] {Resource};
                \node[squarednode] (controller) [below=1.075 of resource] {Controller};
                
                \draw[->, color=uibkorange, very thick] (udp-transport.east) -- (udp-channel.west);
                \draw[->, color=uibkorange, very thick] (tcp-transport.east) -- (tcp-channel.west);
                \draw[->, color=uibkorange, very thick] (udp-channel.east) -| (serializer.north);
                \draw[->, color=uibkorange, very thick] (tcp-channel.east) -| (serializer.south);
                \draw[->, color=uibkorange, very thick] (serializer.east) -- ++(0.2, 0) |- (request-handler.west);
                \draw[<-, color=uibkblue, very thick] (serializer.east) -- ++(0.2, 0) |- (response-handler.west);
                \draw[->, color=uibkorange, very thick] (request-handler.north) -- (router.south);
                \draw[->, color=uibkorange, very thick] (router.east) -| (resource.north);
                \draw[->, color=uibkorange, very thick] (resource.south) -- (controller.north);
                \draw[<-, color=uibkblue, very thick] (response-handler.east) -- (controller.west);
            \end{tikzpicture}
            \caption{Überblick Softwarearchitektur}
            \label{fig:ueberblick-softwarearchitektur}
        \end{figure}
    \end{frame}

    \section{Implementierung}
    \begin{frame}
        \frametitle{Implementierung}

        \begin{itemize}
            \item<1-> \mintinline[breaklines]{csharp}{void CoapMessage Deserialize(ReadonlySpan<byte> value);}
            \item<2-> \mintinline[breaklines]{csharp}{Task<CoapMessage> DeserializeAsync(Stream value, CancellationToken ct);}
            \item<3-> \mintinline[breaklines]{csharp}{void byte[] Serialize(CoapMessage message);}
            \item<4-> \mintinline[breaklines]{csharp}{Task SerializeAsync(Stream stream, CoapMessage message, CancellationToken ct);}
        \end{itemize}
    \end{frame}

    \section{Serialisierung und Deserialisierung}
    \begin{frame}
        \frametitle{Serialisierung und Deserialisierung}
        
        \begin{itemize}
            \item<1-> Laufzeitmessung von \mintinline{csharp}{Serialize}, \mintinline{csharp}{SerializeAsync}, \mintinline{csharp}{Deserialize} und \mintinline{csharp}{DeserializeAsync} mittels \textit{\href{https://github.com/dotnet/BenchmarkDotNet}{BenchmarkDotnet}}.
            \item<2-> Nachrichtengröße durch Optionen und Payload definiert.
            \item<3-> Nachrichten zufällig generiert.
            \item<4-> Messung mehrere hundert Male wiederholt.
            \item<5-> Berechnung Mittelwert aus allen Durchläufen.
            \item<6-> Kennzahlen
            \begin{itemize}
                \item<7-> durchschnittliche Laufzeit
                \item<8-> Fehlertoleranz
                \item<9-> Standardabweichung
                \item<10-> allokierter Speicher
                \item<11-> Anzahl der Objekte in Gen 0 und Gen 1 Speichersegment
            \end{itemize}
        \end{itemize}
    \end{frame}
    \begin{frame}
        \frametitle{Ergebnis}
        
        \begin{figure}[h]
            \centering
            \begin{tikzpicture}[scale=0.8]
                \begin{axis}[
                    ylabel={Laufzeit [\mu s]},
                    ymode=log,
                    grid=both,
                ]
                \addplot[
                    only marks,
                ] table[
                    x expr=\coordindex,
                    col sep=semicolon,
                    y=Mean,
                ]
                    {measurements/deserialize_result.csv};
                \end{axis}
            \end{tikzpicture}
            \caption{Messergebnisse von Deserialize-Methode als Diagramm}
            \label{fig:messergebnisse-deserialize}
        \end{figure}
    \end{frame}
    \begin{frame}
        \frametitle{Ergebnis}
        
        \begin{figure}[h]
            \centering
            \begin{tikzpicture}[scale=0.8]
                \begin{axis}[
                    ylabel={Laufzeit [\mu s]},
                    ymode=log,
                    grid=both,
                ]
                \addplot[
                    only marks,
                ] table[
                    x expr=\coordindex,
                    col sep=semicolon,
                    y=Mean,
                ]
                    {measurements/deserialize_async_result.csv};
                \end{axis}
            \end{tikzpicture}
            \caption{Messergebnisse von DeserializeAsync-Methode als Diagramm}
            \label{fig:messergebnisse-deserialize-async}
        \end{figure}
    \end{frame}
    \begin{frame}
        \frametitle{Ergebnis}
        
        \begin{figure}[h]
            \centering
            \begin{tikzpicture}[scale=0.8]
                \begin{axis}[
                    ylabel={Laufzeit [\mu s]},
                    ymode=log,
                    grid=both,
                ]
                \addplot[
                    only marks,
                ] table[
                    x expr=\coordindex,
                    col sep=semicolon,
                    y=Mean,
                ]
                    {measurements/serialize_result.csv};
                \end{axis}
            \end{tikzpicture}
            \caption{Messergebnisse von Serialize-Methode als Diagramm}
            \label{fig:messergebnisse-seserialize}
        \end{figure}
    \end{frame}
    \begin{frame}
        \frametitle{Ergebnis}
        
        \begin{figure}[h]
            \centering
            \begin{tikzpicture}[scale=0.8]
                \begin{axis}[
                    ylabel={Laufzeit [\mu s]},
                    ymode=log,
                    grid=both,
                ]
                \addplot[
                    only marks,
                ] table[
                    x expr=\coordindex,
                    col sep=semicolon,
                    y=Mean,
                ]
                    {measurements/serialize_async_result.csv};
                \end{axis}
            \end{tikzpicture}
            \caption{Messergebnisse von SerializeAsync-Methode als Diagramm}
            \label{fig:messergebnisse-serialize-async}
        \end{figure}
    \end{frame}

    \section{Nachrichtenübertragung}
    \begin{frame}
        \frametitle{Nachrichtenübetragung}
    
        \begin{itemize}
            \item<1-> Simulation einer langsamen Client-Server-Kommunikation.
            \item<2-> Geschwindigkeit von 4 B/s.
            \item<3-> Nachrichtengröße durch Anzahl der Optionen und Größe der Payload definiert.
            \item<4-> Client und Server auf selbem System.
            \item<5-> Kommunikation über \textit{localhost} (127.0.0.1) mittels TCP.
            \item<6-> Server öffnet zwei Ports
            \begin{itemize}
                \item<7-> 5683: Asynchrone Verarbeitung.
                \item<8-> 5684: Synchrone Verarbeitung.
            \end{itemize}
            \item<9-> Nachricht von Client an Server.
            \item<10-> Fünf voneinander unabhängige Durchläufe.
            \item<11-> Laufzeit: Start bis Ende der Übertragung.
        \end{itemize}
    \end{frame}

    \begin{frame}
        \frametitle{Ergebnis}
    
        \begin{itemize}
            \item<1-> Asynchrone Verarbeitung skaliert mit steigender Nachrichtengröße.
            \item<2-> Differenz zwischen Synchron und Asynchron nur einige hundert Millisekunden bis Sekunden.
        \end{itemize}
    \end{frame}

    \section{Fazit}
    \begin{frame}
        \frametitle{Fazit}
    
        \begin{itemize}
            \item<1-> Einsatz von Asynchronität abhängig von Problemstellung \textbf{\textcolor{uibkblue}{- allgemein jedoch die bessere Wahl}}.
            \item<2-> Verarbeitungsgeschwindigkeit von Nachrichtengröße abhängig.
            \item<3-> CoAP scheint nicht von Asynchronität zu profitieren.
        \end{itemize}
    \end{frame}

    \section{Ausblick}
    \begin{frame}
        \frametitle{Ausblick}
    
        \begin{itemize}
            \item<1-> Implementierung von \textbf{\textcolor{uibkblue}{retransmission}} und \textbf{\textcolor{uibkblue}{block-wise transfer}}.
            \item<2-> Implementierung \textbf{\textcolor{uibkblue}{CoAP over TCP, Websockets und TLS}}.
            \item<3-> Verbesserung der Nachrichtenverarbeitung.
            \item<4-> Implementation von einigen \textit{syntactic sugar} Funktionalitäten aus ASP.NET Core.
        \end{itemize}
    \end{frame}

\end{document}