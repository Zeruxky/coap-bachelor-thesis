\documentclass[11pt,t,usepdftitle=false,aspectratio=169,usenames,dvipsnames]{beamer}
\usetheme[nototalframenumber,foot,logo,nosectiontitlepage]{uibk}
\headerimage{1}

\usepackage{lmodern}
\usepackage{xcolor}
\usepackage{minted}
\usepackage{tikz}
\usepackage{hyperref}
\usetikzlibrary{calc}

\title[Initial presentation CoAP.NET]{Increasing throughput of server applications by using asynchronous techniques}
\subtitle{A case study on CoAP.NET}
\author{Philip Wille\\Supervisors: Michael Felderer, Andreas Danek}
\date{2021-01-12}

\begin{document}
    \maketitle
    \begin{frame}
        \vspace*{1cm plus 1fil}
        \tableofcontents
        \vspace*{0cm plus 1fil}
    \end{frame}
    
    \section{Programming paradigms}
    \begin{frame}
        \frametitle{Programming paradigms}
        \begin{itemize}
            \item<1-> Synchronous
            \begin{itemize}
                \item<3-> Must \textcolor{uibkblue}{\textbf{stop}} program flow.
                \item<5-> \textcolor{uibkblue}{\textbf{Checks}} periodically.
                \item<7-> Marked as \textcolor{uibkblue}{\textbf{Blocked}} (Linux) or \textcolor{uibkblue}{\textbf{Waiting}} (Windows).
            \end{itemize}
            \item<2-> Asynchronous
            \begin{itemize}
                \item<4-> Can \textcolor{uibkblue}{\textbf{go further}} in program flow.
                \item<6-> Will be \textcolor{uibkblue}{\textbf{notified}} by event.
                \item<8-> \textcolor{uibkblue}{\textbf{Free}} for other tasks.
            \end{itemize}
        \end{itemize}
    \end{frame}

    \subsection{Ordering a book in synchronous and asynchronous way}
    \begin{frame}
        \frametitle{Ordering a book}
        \begin{itemize}
            \item<1-> Synchronous way
            \begin{enumerate}
                \item<2-> You are ordering \textit{Clean Code (Robert C. Martin)} from amazon.com
                \item<3-> You are sitting on the couch and waiting for the book.
                \item<4-> Postman is knocking on your door and giving you the book.
                \item<5-> You start reading it.
                \item<6-> You have finished it.
                \item<7-> You are going outside, meeting friends, go hiking and so on.
            \end{enumerate}
            \item<8-> Asynchronous way
            \begin{enumerate}
                \item<9-> You are ordering \textit{Clean Code (Robert C. Martin)} from amazon.com
                \item<10-> You are going outside, meeting friends, go hiking and so on.
                \item<11-> In the meanwhile the postman delivers the book to your home.
                \item<12-> You are coming home and picking up the book.
                \item<13-> You start reading it.
                \item<14-> You have finished it.
            \end{enumerate}
        \end{itemize}
    \end{frame}

    \section{Synchronous and asynchronous server}
    \subsection{Synchronous server}
    \begin{frame}
        \frametitle{Synchronous server}
        \begin{figure}[ht]
            \begin{tikzpicture}
                \node at (0,.3) {Database};
                \node at (-3,.3) {Server};
                \node at (-6,.3) {Client 2};
                \node at (-9,.3) {Client 1};
                
                \draw[thick, uibkorange] (-9,0) -- node[left] {prepare}(-9,-0.5);
                \draw[thick, uibkgraym] (-6,0) -- (-6,-0.5);
                \draw[thick, uibkgraym] (-3,0) -- (-3,-0.5);
                \draw[thick, uibkgraym] (-0,0) -- (0,-0.5);
                \onslide<1->
                \draw[->, thick, uibkorange] (-9,-0.5) -- node[midway,above] {Request 1} (-3,-0.5);
                \onslide<2->
                \draw[thick, uibkorange] (-3,-0.5) -- node[left] {prepare SQL}(-3,-1);
                \onslide<3->
                \draw[->, thick, uibkorange] (-3,-1) -- node[midway,above] {SQL Request 1} (-0,-1);
                \draw[thick, uibkgraym] (0, -0.5) -- (0, -1);
                \onslide<4->
                \draw[thick, uibkorange] (-0,-1) -- node [right] {execute} (0,-3);
                \onslide<5->
                \draw[thick, uibkgraym] (-6, -0.5) -- (-6, -1);
                \draw[thick, uibkblue] (-6, -1) -- node[left] {prepare} (-6, -1.5);
                \onslide<6->
                \draw[->, thick, uibkblue] (-6,-1.5) -- node[midway,above] {Request 2} (-3,-1.5);
                \draw[thick, uibkgraym] (-3,-1) -- (-3,-1.5);
                \onslide<7->
                \draw[<-, thick, uibkorange] (-3,-3) -- node [midway, above] {SQL Response 1} (0,-3);
                \draw[thick, uibkgraym] (-3,-1.5) -- (-3,-3);
                \onslide<8->
                \draw[thick, uibkorange] (-3,-3) -- (-3,-3.5);
                \draw[<-, thick, uibkorange] (-9,-3.5) -- node [midway, above] {Response 1} (-3,-3.5);
                \draw[thick, uibkgraym] (-9,-0.5) -- (-9,-3.5);
                \onslide<9->
                \draw[thick, uibkorange] (-9,-3.5) -- node [midway, left] {process} (-9,-5.5);
                \onslide<10->
                \draw[thick, uibkblue] (-3,-3.5) -- node [midway, left] {prepare SQL} (-3,-4);
                \onslide<11->
                \draw[thick, uibkgraym] (-0,-3) -- (-0,-4);
                \draw[->, thick, uibkblue] (-3,-4) -- node [midway, above] {SQL Request 2} (-0,-4);
                \onslide<12->
                \draw[thick, uibkblue] (-0,-4) -- node [midway, right] {execute} (-0,-5);
                \onslide<13->
                \draw[thick, uibkgraym] (-3,-4) -- (-3,-5);
                \draw[<-, thick, uibkblue] (-3,-5) -- node [midway, above] {SQL Response 2} (-0,-5);
                \onslide<14->
                \draw[thick, uibkblue] (-3,-5) -- (-3,-5.5);
                \onslide<15->
                \draw[thick, uibkgraym] (-6,-1.5) -- (-6,-5.5);
                \draw[<-, thick, uibkblue] (-6,-5.5) -- node [midway, above] {Response 2} (-3,-5.5);
                \draw[thick, uibkgraym] (0, -5) -- (0, -5.5);
            \end{tikzpicture}
            \caption{Sequence diagram of synchronous server}
            \label{figure:sequence-diagram-of-synchronous-server}
        \end{figure}
    \end{frame}

    \subsection{Asynchronous server}
    \begin{frame}
        \frametitle{Asynchronous server}
        \begin{figure}[ht]
            \begin{tikzpicture}
                \node at (0,.3) {Database};
                \node at (-3,.3) {Server};
                \node at (-6,.3) {Client 2};
                \node at (-9,.3) {Client 1};
                
                \draw[thick, uibkorange] (-9,0) -- node[left] {prepare}(-9,-0.5);
                \draw[thick, uibkgraym] (-6,0) -- (-6,-0.5);
                \draw[thick, uibkgraym] (-3,0) -- (-3,-0.5);
                \draw[thick, uibkgraym] (-0,0) -- (0,-0.5);
                \onslide<1->
                \draw[->, thick, uibkorange] (-9,-0.5) -- node[midway,above] {Request 1} (-3,-0.5);
                \onslide<2->
                \draw[thick, uibkorange] (-3,-0.5) -- node[left] {prepare SQL}(-3,-1);
                \onslide<3->
                \draw[thick, uibkgraym] (-6,-0.5) -- (-6,-1.5);
                \draw[->, thick, uibkorange] (-3,-1) -- node[midway,above] {SQL Request 1} (-0,-1);
                \draw[thick, uibkgraym] (0, -0.5) -- (0, -1);
                \onslide<4->
                \draw[thick, uibkorange] (-0,-1) -- node [right] {execute} (0,-3);
                \onslide<5->
                \draw[thick, uibkgraym] (-3,-1) -- (-3,-1.5);
                \draw[->, thick, uibkblue] (-6,-1.5) -- node[midway,above] {Request 2} (-3,-1.5);
                \onslide<6->
                \draw[thick, uibkblue] (-3,-1.5) -- node[midway, left] {prepare SQL} (-3,-2);
                \onslide<7->
                \draw[->, thick, uibkblue] (-3,-2) -- node [midway, above] {SQL Request 2} (-0,-2);
                \onslide<8->
                \draw[thick, uibkgraym] (-3,-2) -- (-3,-3);
                \draw[<-, thick, uibkorange] (-3,-3) -- node [midway, above] {SQL Response 1} (0,-3);
                \onslide<9->
                \draw[thick, uibkorange] (-3,-3) -- (-3,-3.5);
                \draw[<-, thick, uibkorange] (-9,-3.5) -- node [midway, above] {Response 1} (-3,-3.5);
                \draw[thick, uibkgraym] (-9,-0.5) -- (-9,-3.5);
                \onslide<10->
                \draw[thick, uibkblue] (-0,-3) -- node [midway, right] {execute} (-0,-4);
                \onslide<11->
                \draw[thick, uibkorange] (-9,-3.5) -- node [midway, left] {process} (-9,-5.5);
                \onslide<12->
                \draw[thick, uibkgraym] (-3,-3.5) -- (-3,-4);
                \draw[<-, thick, uibkblue] (-3,-4) -- node [midway, above] {SQL Response 2} (-0,-4);
                \onslide<13->
                \draw[thick, uibkblue] (-3,-4) -- (-3,-4.5);
                \draw[thick, uibkgraym] (0,-4) -- (0,-5.5);
                \draw[<-, thick, uibkblue] (-6,-4.5) -- node [midway, above] {Response 2} (-3,-4.5);
                \draw[thick, uibkgraym] (-6,-1.5) -- (-6,-4.5);
                \onslide<14->
                \draw[thick, uibkblue] (-6,-4.5) -- node [midway, left] {process} (-6,-5.5);
                \draw[thick, uibkgraym] (-3,-4.5) -- (-3,-5.5);
            \end{tikzpicture}
            \caption{Sequence diagram of asynchronous server}
            \label{figure:sequence-diagram-of-asynchronous-server}
        \end{figure}
    \end{frame}

    \section{Task-based Asynchronous Pattern (TAP)}
    \begin{frame}
        \frametitle{Task-based Asynchronous Pattern (TAP)}
        \begin{itemize}
            \item<1-> Developed by \textcolor{uibkblue}{\textbf{Microsoft}}.
            \item<2-> Easy transformation \textcolor{uibkblue}{\textbf{Synchronous Code $\rightarrow$ Asynchronous Code}}.
            \item<3-> \textcolor{uibkblue}{\textbf{Built-in}} in C\#.
            \item<4-> Main components
            \begin{itemize}
                \item<5-> Task
                \item<6-> Task<TResult>
                \item<7-> CancellationToken
                \item<8-> async/await keyword
            \end{itemize}
        \end{itemize}
    \end{frame}

    \begin{frame}
        \frametitle{Synchronous execution in C\#}
        
        \begin{listing}[H]
            \inputminted[framesep=2mm, baselinestretch=1.2, fontsize=\scriptsize, linenos]{csharp}{codes/example_synchronous.cs}
            \caption{Synchronous usage in C\#}
            \label{listing:synchronous-usage-in-csharp}
        \end{listing}
    \end{frame}

    \begin{frame}
        \frametitle{Event-based execution in C\#}
        
        \begin{listing}[H]
            \inputminted[framesep=2mm, baselinestretch=1.2, fontsize=\scriptsize, linenos]{csharp}{codes/example_eventbased.cs}
            \caption{Usage of events in C\#}
            \label{listing:usage-of-events-in-csharp}
        \end{listing}
    \end{frame}

    \begin{frame}
        \frametitle{Asynchronous execution in C\#}

        \begin{listing}[H]
            \inputminted[framesep=2mm, baselinestretch=1.2, fontsize=\scriptsize, linenos]{csharp}{codes/example_asynchronous.cs}
            \caption{Asynchronous usage in C\#}
            \label{listing:asynchronous-usage-in-csharp}
        \end{listing}
    \end{frame}

    \section{Constrained Application Protocol (CoAP)}
    \begin{frame}
        \frametitle{Constrained Application Protocol (CoAP)}
        \begin{itemize}
            \item<1-> Defined in \href{https://tools.ietf.org/html/rfc7252}{RFC 7252}.
            \item<2-> Designed for \textcolor{uibkblue}{\textbf{constrained}} environments.
            \item<3-> \textcolor{uibkblue}{\textbf{Request/response}} interaction model.
            \item<4-> Uses \textcolor{uibkblue}{\textbf{U}}ser \textcolor{uibkblue}{\textbf{D}}atagram \textcolor{uibkblue}{\textbf{P}}rotocol (UDP).
            \item<5-> Implementation for several programming languages.
        \end{itemize}
    \end{frame}

    \subsection{Example of CoAP message}
    \begin{frame}
        \frametitle{Example of CoAP message}
        \textcolor{uibkblue}{\textbf{01}}\textcolor{uibkorange}{\textbf{01}}\textcolor{uibkblue}{\textbf{0100}}\textcolor{uibkorange}{\textbf{01000101}}\textcolor{uibkblue}{\textbf{1101111100011001}}\textcolor{uibkorange}{\textbf{00000000000000001110111110001110}}\textcolor{uibkblue}
        \newline {\textbf{....}}\textcolor{uibkorange}{\textbf{11111111}}\textcolor{uibkblue}{\textbf{00000010}}
        \begin{itemize}
            \item \textcolor{uibkblue}{\textbf{Version}}: 1 (01......)
            \item \textcolor{uibkorange}{\textbf{Type}}: Non-Confirmable (..01....)
            \item \textcolor{uibkblue}{\textbf{Token Length}}: 4 (....0100)
            \item \textcolor{uibkorange}{\textbf{Code}}: 2.05 Content (01000101)
            \item \textcolor{uibkblue}{\textbf{Message ID}}: 51773 (11011111 00011001; Big endian)
            \item \textcolor{uibkorange}{\textbf{Token}}: 61326 (00000000 00000000 11101111 10001110)
            \item \textcolor{uibkblue}{\textbf{Options}}: Set of options
            \item \textcolor{uibkorange}{\textbf{Payload marker}}: 255 (11111111)
            \item \textcolor{uibkblue}{\textbf{Payload}}: 2 (00000010)
        \end{itemize}
    \end{frame}

    \subsection{CoAP.NET}
    \begin{frame}
        \frametitle{CoAP.NET}
        \begin{itemize}
            \item<1-> Implementation of CoAP for C\#.
            \item<2-> Development inactive.
            \item<3-> Partially asynchronous.
            \item<4-> Offers a client and server implementation.
        \end{itemize}
    \end{frame}

    \section{Bachelor thesis}
    \begin{frame}
        \frametitle{Bachelor thesis}

        \begin{itemize}
            \item<1-> \textcolor{uibkblue}{\textit{>>Has an asynchronous implementation of a server an impact on its throughput?<<}}
            \item<2-> Fully rewrite CoAP.NET library \onslide<3->\textbf{\textcolor{uibkblue}{except retransmission and block-wise transfer}}. 
            \item<4-> Implement tests for measuring throughput.
            \item<5-> Compare synchronous with asynchronous version.
            \item<6-> Source code freely available at \href{https://github.com/world-direct/CoAP.NET}{GitHub}.
            \item<7-> Collaboration with World-Direct eBusiness solutions GmbH.
        \end{itemize}
        \onslide<8->
        \begin{figure}[ht]
            \begin{tikzpicture}
                \draw[step=2, uibkgraym, very thin] (0, 0) grid (8,-1);
                \node at (0,.6) {12.01.2021};
                \node at (2,-1.6) {04.2021};
                \node at (4,.6) {06.2021};
                \node at (6,-1.6) {07.2021};
                \node at (8,.6) {09.2021};
                \node at (10,-1.6) {10.2021};
                \foreach \x in {0,2,4,6,8}
                    \fill[fill=uibkblue] (\x, 0) rectangle (\x+2, -1);
                \foreach \y in {0, 2, 4, 6, 8, 10}
                    \draw[thick, uibkgraym] (\y, .3) -- (\y, -1.3);
                \node[text=uibkorange] at (1,-.5) {M1};
                \node[text=uibkorange] at (3,-.5) {M2};
                \node[text=uibkorange] at (5,-.5) {M3};
                \node[text=uibkorange] at (7,-.5) {M4};
                \node[text=uibkorange] at (9,-.5) {M5};
            \end{tikzpicture}
            \caption{Phase-Milestone plan}
            \label{figure:phase-milestone-plan}
        \end{figure}
    \end{frame}

    \begin{frame}
        \frametitle{Bachelor thesis}
        \begin{figure}[ht]
            \begin{tikzpicture}
                \draw[step=2, uibkgraym, very thin] (0, 0) grid (8,-1);
                \node at (0,.6) {12.01.2021};
                \node at (2,-1.6) {04.2021};
                \node at (4,.6) {06.2021};
                \node at (6,-1.6) {07.2021};
                \node at (8,.6) {09.2021};
                \node at (10,-1.6) {10.2021};
                \foreach \x in {0,2,4,6,8}
                    \fill[fill=uibkblue] (\x, 0) rectangle (\x+2, -1);
                \foreach \y in {0, 2, 4, 6, 8, 10}
                    \draw[thick, uibkgraym] (\y, .3) -- (\y, -1.3);
                \node[text=uibkorange] at (1,-.5) {M1};
                \node[text=uibkorange] at (3,-.5) {M2};
                \node[text=uibkorange] at (5,-.5) {M3};
                \node[text=uibkorange] at (7,-.5) {M4};
                \node[text=uibkorange] at (9,-.5) {M5};
            \end{tikzpicture}
            \caption{Phase-Milestone plan}
            \label{figure:phase-milestone-plan}
        \end{figure}
        \onslide<1->
        \begin{enumerate}
            \item<2-> 12.01.2021: Initial presentation.
            \item<3-> 04.2021: Finish asynchronous implementation.
            \item<4-> 06.2021: Finish measurements.
            \item<5-> 07.2021: Finish comparison.
            \item<6-> 09.2021: Finish writing of bachelor thesis.
            \item<7-> 10.2021: Final presentation.
        \end{enumerate}
    \end{frame}
\end{document}