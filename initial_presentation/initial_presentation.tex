\documentclass[xcolor=dvipsnames]{beamer}
\usepackage[utf8]{inputenc}
\usepackage{hyperref}
\usepackage{xcolor}
\usepackage[autostyle]{csquotes}
\usepackage[
    backend=biber,
    style=authoryear-icomp,
    sortlocale=de_DE,
    natbib=true,
    url=false, 
    doi=true,
    eprint=false
]{biblatex}
\addbibresource{references.bib}
\setbeamertemplate{bibliography entry title}{}
\setbeamertemplate{bibliography entry location}{}
\setbeamertemplate{bibliography entry note}{}
\usetheme{Szeged}
\usecolortheme{beaver}
\defbeamertemplate*{title page}{customized}[1][]
{
  \usebeamerfont{title}\inserttitle\par
  \usebeamerfont{subtitle}\usebeamercolor[fg]{subtitle}\insertsubtitle\par
  \bigskip
  \usebeamerfont{author}\insertauthor\par
  \usebeamerfont{institute}\insertinstitute\par
  \usebeamerfont{date}\insertdate\par
  \usebeamercolor[fg]{titlegraphic}\inserttitlegraphic
}
\title{Asynchronous vs. synchronous programming}
\subtitle{Is it worth to await?}
\author{Philip Wille}
\institute[UIBK]
{
    \inst{1}%
    Fakultät Informatik\\
    Universität Innsbruck
}
\date{\today}
\begin{document}
    \frame{\maketitle}

    \begin{frame}
        \frametitle{Table of contents}
        \tableofcontents
    \end{frame}

    \section{Introduction}
    \begin{frame}
        \frametitle{Introduction}
        \begin{itemize}
            \item A program interacts with many other systems, like databases, web services or file systems.
            \item A program that requests data from an external system must wait for the response.
            \item The time interval between sending a request and getting the corresponding response can take a long time.
            \item Without optimization the program is doing busy waiting and blocks therefore the main thread.
            \item This results in an unresponsive program. Therefore, for example a program with a graphical user interface (GUI) becomes unresponsive or can not provide a high throughput of requests.
        \end{itemize}
    \end{frame}

    \section{Background}
    \begin{frame}
        \frametitle{Background}
        \begin{itemize}
            \item Two main paradigms of programming: \alert{synchronous programming} and \alert{asynchronous programming}.
            \item Synchronous programming: \alert{Waiting for the termination} of an action. The \alert{main thread is blocked} in the meantime, because of busy waiting.
            \item Asynchronous programming: \alert{Waiting for an event} that represents the termination of an action and \alert{notifies} the program to proceeding in the work flow. The \alert{main thread} can work on \alert{other tasks} in the meantime.
        \end{itemize}
    \end{frame}

    \section{Technical work}
    \begin{frame}
        \frametitle{Technical work}
        \begin{itemize}
            \item Research question: Does benefit a server implementation from asynchronous programming?
            \item Trying to answer this question based on a proof of concept of \href{https://github.com/smeshlink/CoAP.NET}{CoAP.NET}.
            \item CoAP.NET is an open source C\# implementation of the \textbf{Co}nstrained \textbf{A}pplication \textbf{P}rotocol (\href{https://tools.ietf.org/html/rfc7252}{CoAP}).
            \item \textit{"The Constrained Application Protocol (CoAP) is a specialized web transfer protocol for use with constrained nodes and constrained (e.g., low-power, lossy) networks."} (\cite{RFC_7252_CoAP})
        \end{itemize}
    \end{frame}
    \section{Milestones}
    \begin{frame}
        \frametitle{Milestones}
        \begin{table}[]
            \begin{tabular}{ll}
                & \textbf{Description} \\ \hline
                \textbf{Milestone 1} & Rewrite to .NET Standard 2.0 \\ \hline
                \textbf{Milestone 2} & Implement asynchronous pattern \\ \hline
                \textbf{Milestone 3} & Specification from World-Direct \\ \hline
                \textbf{Milestone 4} & Benchmark synchronous and asynchronous version \\ \hline
                \textbf{Milestone 5} & Releasing V1.0 \\ \hline
            \end{tabular}
        \end{table}
    \end{frame}
\end{document}